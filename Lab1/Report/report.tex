\documentclass{article}
    \author{keith Null}
    \title{Report}
    \date{\today}
    \usepackage{minted}
    \usepackage{xcolor}
    \definecolor{bg}{rgb}{0.95,0.95,0.95}
    \usemintedstyle{manni}
    \setminted{
    linenos,
    autogobble,
    breaklines,
    breakautoindent,
    bgcolor=bg,
    numberblanklines=false,
    }

\begin{document}
    \maketitle
    \section{Problem Description}
In  exercise  one,  the  task  is  to  create  a  database  and  tables
according to the given data. And then, use Python to store the data
into the database.

In exercise two, the task is to design SQL enquiries to get the
data we need from the database.

    \section{Problem Analysis}
To create the database and tables, it’s better to operate directly
in the MySQL shell rather than using Python. So we just need to run
codes  like  “CREATE  DATABASE  xxx”  and  “CREATE  TABLE  xxx”  in  the
shell.

To store data, evidently we should use Python to connect to the
database and execute SQL enquiries like “INSERT INTO”.

As  for  designing  SQL  enquiries  to  get  data  from  the  database,
“SELECT xxx FROM xxx” and “JOIN” are practical ways.

    \section{Experiment Process}
        \subsection{Preparpations}
Although only MySQL is needed in this experiment, for the sake of
simplicity, we install XAMPP (Version 7.2.2) and then start MySQL in
the panel to get everything prepared properly.
        \subsection{Creating the database and tables}
In the shell of MySQL, enter codes as follows:
            \begin{minted}{bash}
Mysql -u root
            \end{minted}
Create  the  database  and  one  thing  to  note,  to  avoid  encoding
problems, we need to set the default character set as utf8:
            \begin{minted}{SQL}
CREATE DATABASE AcademicDB DEFAULT CHARSET utf8;
            \end{minted}
Then use the database we have created and create tables in it. When
creating tables, we need to set the data type in accordance with the
data and also use utf8 to encode.
            \begin{minted}{sql}
USE AcademicDB;
CREATE  TABLE  papers  (PaperID  char(8),  Title  text,PaperPublishYear integer(4), ConferenceID char(8)) DEFAULT charset utf8;
CREATE  TABLE  authors  (AuthorID  char(8),  AuthorName tinytext) DEFAULT charset utf8;
CREATE  TABLE   conferences   (ConferenceID   char(8),ConferenceName tinytext)DEFAULT charset utf8;
CREATE  TABLE   affiliations   (AffliationID   char(8),AffliationName tinytext)DEFAULT charset utf8;
CREATE TABLE paper_author_affiliation (PaperID char(8),AuthorID  char(8),  AffliationID  char(8),AuthorSequence tinyint unsigned)DEFAULT charset utf8;
CREATE TABLE   paper_reference   (PaperID   char(8),ReferenceID char(8))DEFAULT charset utf8;
            \end{minted}
        \subsection{Using Python to save the data in the database}
To simplify and beautify the code, we import the module PyMySQL and define  a  class  DataInserter  with  methods  like  connect()  and insert\_from\_file().

First, after creating an instance of this class, we need to connect
it the database by providing host, port, user and password, etc.
            \begin{minted}{python}
def connect(self, user, password, db,host="localhost", port=3306, charset="utf8"):
    try:
        self.connection = pymysql.connect(
        host=host, user=user, password=password,
        db=db, port=port, charset=charset)
        self.cursor = self.connection.cursor()
    except:
        print("Failed to connect to the database!")
    else:
        self.connected = True
            \end{minted}
When insert\_from\_file() is called, it checks whether the connection
has been established successfully and if not, just return directly.

In addition, to keep the user informed of the process progress, it
opens  the  file  twice to  count  the  lines  first  and  then  handle  it
actually.

Note that as some data contain digital types, we need to convert
some string into integer. For instance, what we need is 2004 rather
than  “2004”.  To  implement  this  and  avoid  unnecessary  code,  this
method takes to\_digit (a tuple) as a parameter to mark which columns
need to be converted.
            \begin{minted}{python}
def insert_from_file(self, table,file_name,to_digit=()):
    if not self.connected:
        print("Haven't connected to the database yet!")
        return
    all_line = 0
    with  open(file_name,  encoding="utf8",  mode="r")  as  file:
        for line in file:
            all_line += 1
    print("{0} has {1} line{2} to handle.".format(
       file_name, all_line, "s" if all_line >= 2 else ""))
    with open(file_name,  encoding="utf8",  mode="r") as file:
        current_line = 1
        for line in file:
            print("{0}/{1}".format(current_line, all_line), end="\r")
            data = line.strip().split("\t")
            for i in to_digit:
                data[i] = int(data[i])
            sql="INSERT INTO  {0}  VALUES  {1};".format(table,tuple(data))
            try:
                self.cursor.execute(sql)
            except:
                print("Failed  to  insert  data  {0}  into  table   {1}(Line   {2}   of   {3})".format(data,table,current_line , file_name))
                self.connection.rollback()
                break
            current_line += 1
        self.connection.commit()
    print("Done successfully!")
            \end{minted}
    For example, when saving papers.txt into papers table, we just need
to call the method in this way:
            \begin{minted}{python}
inserter.insert_from_file(table="papers", file_name="data/papers.txt", to_digit=(2,))
            \end{minted}
        \subsection{Designing SQL enquiries}
Without  too  much  difficulty,  the  SQL  enquires  satisfying  the
requirements are as follows:
            \begin{minted}{sql}
SELECT title,paperpublishyear
FROM papers
WHERE paperid="58EA85EE";

SELECT authorid
FROM paper_author_affiliation
WHERE paperid="58EA85EE"
ORDER BY AuthorSequence ASC;

SELECT authors.authorname
FROM authors
INNER JOIN paper_author_affiliation ON authors.authorid
=paper_author_affiliation.authorid
WHERE paper_author_affiliation.paperid="58EA85EE"
ORDER BY paper_author_affiliation.AuthorSequence ASC;

SELECT count(*)
FROM paper_reference
WHERE referenceid="800F1DB6";
            \end{minted}
        \subsection{Optimizing the enquiry time with index}
Running the above enquiries might not take a very long time, even
the slowest one of them (to be exact, the third one) only takes a few
seconds. But to pursue the best, we can speed the enquiries up using
index.

Take the third enquiry which involves “JOIN” as example, without
any optimization, it takes 2.54s. If we simply create an index in the
table  paper\_author\_affiliation,  the  same  enquiry  only  takes  0.24s,
faster more than 10 times.
            \begin{minted}{sql}
CREATE INDEX index_paperid ON paper_author_affiliation(paperid);
            \end{minted}
So obviously, creating index properly can definitely optimize the
enquiry time.
    \section{Conclusion and reflection}
If  the  data  involve  non-ascii  characters,  remember  to  set  the
character set as utf8.

Use  class  and  module  to  write  Python  code  effectively  and
beautifully.

If some SQL enquiries are too slow, try to create index properly.
\end{document}
